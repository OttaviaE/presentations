%-------------------------
% Resume in Latex
% Author : Shubhi Rani
% License : MIT
%------------------------

\documentclass[letterpaper,12pt]{article}

\usepackage{latexsym}
\usepackage[empty]{fullpage}
\usepackage{titlesec}
\usepackage{marvosym}
\usepackage[usenames,dvipsnames]{color}
\usepackage{verbatim}
\usepackage{enumitem}
\usepackage[pdftex]{hyperref}
\usepackage{fancyhdr}
\usepackage[utf8]{inputenc}
\usepackage{lastpage}
\pagenumbering{alph}

\pagestyle{fancy}
\fancyhf{} % clear all header and footer fields
\fancyfoot{}
\renewcommand{\headrulewidth}{0pt}
\renewcommand{\footrulewidth}{0pt}
\pagenumbering{arabic}
\cfoot{Page \thepage\ of \pageref{LastPage}}
% Adjust margins
\addtolength{\oddsidemargin}{-0.375in}
\addtolength{\evensidemargin}{.90in}
\addtolength{\textwidth}{1in}
\addtolength{\topmargin}{-.5in}
\addtolength{\textheight}{1in}

\urlstyle{rm}

\raggedbottom
\raggedright
\setlength{\tabcolsep}{0in}

% Sections formatting
\titleformat{\section}{
  \vspace{-3pt}\scshape\raggedright\large
}{}{0em}{}[\color{black}\titlerule \vspace{-5pt}]

%-------------------------
% Custom commands
\newcommand{\resumeItem}[2]{
  \item\small{
    \textbf{#1}{: #2 \vspace{-2pt}}
  }
}

\newcommand{\resumeItemWithoutTitle}[1]{
  \item\small{
    {\vspace{-2pt}}
  }
}

\newcommand{\resumeSubheading}[4]{
  \vspace{-1pt}\item
    \begin{tabular*}{0.97\textwidth}{l@{\extracolsep{\fill}}r}
      \textbf{#1} & #2 \\
      \textit{\small#3} & \textit{\small #4} \\
    \end{tabular*}\vspace{-5pt}
}


\newcommand{\resumeSubItem}[2]{\resumeItem{#1}{#2}\vspace{-4pt}}

\renewcommand{\labelitemii}{$\circ$}

\newcommand{\resumeSubHeadingListStart}{\begin{itemize}[leftmargin=*]}
\newcommand{\resumeSubHeadingListEnd}{\end{itemize}}
\newcommand{\resumeItemListStart}{\begin{itemize}}
\newcommand{\resumeItemListEnd}{\end{itemize}\vspace{-5pt}}

%-------------------------------------------
%%%%%%  CV STARTS HERE  %%%%%%%%%%%%%%%%%%%%%%%%%%%%


\begin{document}

%----------HEADING-----------------
\begin{tabular*}{\textwidth}{l@{\extracolsep{\fill}}r}
  \textbf{{\LARGE Ottavia M. Epifania}} & Email : \href{otta.epifania@gmail.com}{otta.epifania@gmail.com}\\
  \href{https://github.com/OttaviaE}{GitHub: https://github.com/OttaviaE} &  Phone: +39-3479261377 \\
  \href{https://osf.io/profile/}{Open Science Framework: https://osf.io/profile/} & Birth Date: January 10\textsuperscript{th}, 1990\\
\end{tabular*}

\section{Current Employment}
\resumeSubHeadingListStart
\resumeSubheading
  {Temporary research fellow}{Padova, IT}{University of Padova}
{July 2022 - June 2024}
 
 {\small PRIN 2020 Project: ``Computerized, Adaptive and Personalized Assessment of Executive Functions and Fluid Intelligence''}
 
  {\small Principal Investigator: Professor Luca Stefanutti}
  
    {\small Law:   art. 22   della Legge 240/2010}
    
        {\small Note:   Currently underway}
  

 
\resumeSubheading
 {Lecturer}{Milan, IT}{60-hour course (First semester), 8 c.f.u.}
{Since Academic Year 2021/2022}

{\small Course: \textbf{Statistical Psychometrics} (Introduction to Measurement Theory in Psychology, univariate and bivariate descriptive statistics, Probability Theory, Statistical Inference)}

{\small School: Bachelor degree in Psychology, Catholic University of the Sacred Heart, Milan, Italy}

 {\small Note:  Currently underway, Confirmed for the Academic Year 2023/2024}

\resumeSubheading
{Review Editor}{}{Frontiers in Quantitative Psychology}
{July 2021 - On going}

\resumeSubHeadingListEnd

%-----------EDUCATION-----------------
\section{Education}
\resumeSubHeadingListStart
\resumeSubheading
{Ph.D. in Psychological Sciences; cum laudae}{Padova, IT}{University of Padova}
{Oct 2017 - May 2021}

{\small Thesis defended on May, 14\textsuperscript{th}, 2021}

{\small Thesis reviewers: Prof. David Andrich, Prof. Francis Tuerlinckx. }

{\small Defense committee: Prof. Luca Andrighetto, Prof. Caterina Primi, Prof. Katie Wolsiefer}

{\small Thesis: Inglorious Measures: A Linear Mixed-Effects Model approach for a Rasch analysis of implicit measure accuracy and time responses}

{\small Advisor: Professor Egidio Robusto \\ Co-advisor: Professor Gianmarco Altoè}

{\small Attended courses: Null hypothesis testing, \texttt{R} basics/advanced, Inferential Statistics (NHST), Inferential Statistics (Bayesian), Latent trait modeling, Scientific writing.}



\resumeSubheading
{Visiting research scholar}{Ohio, USA}{The Ohio State University}
{Jan 2019 - May 2019}

{\small Collaboration with Prof. De Boeck (i.e., multilevel data modeling, latent variable modeling)}

\resumeSubheading{Master Degree in Psychology, 110/110 cum laude}{Genova, IT}{University of Genova}
{Nov 2012 - Nov 2014}

{\small Supervisor: Prof. Carlo Chiorri}

{\small Thesis: Pratiche di Body Modifications: Associazione con la tendenza al self injury e con i tratti di personalità}

\resumeSubheading{Bachelor Degree in Psychology, 109/110}{Genova, IT}{University of Genova}
{Oct 2012 - Oct 2009}

{\small Supervisor: Dr. Paola Cardinali}

{\small Thesis: Violenza e processi Psicosociali: Il G8 di Genova}

\resumeSubHeadingListEnd

% previous work experience 
\section{Previous work experience}
\resumeSubHeadingListStart
\resumeSubheading
{Temporary research fellow}{Padova, IT}
{University of Padova}{Feb 2021 - June 2022}

{\small Project: ``New frontiers in developing tests for Psychological Assessment''}

{\small Principal Investigator: Professor Pasquale Anselmi}

    {\small Law:   art. 22   della Legge 240/2010}

{\small Note: 5-month mandatory maternity leave, from 24\textsuperscript{th} December 2021 to 24\textsuperscript{th} May 2022}


\resumeSubheading
{Translator}{Busto Arsizio, IT}
{ EduVal Project srl}{Sep 2018 - Dec 2018}

{\small English-Italian translations of documents, reports, tests for the OECD-SSES project}


\resumeSubheading
{Temporary research fellow}{Genova, IT}
{Educational Technologies Institute - National Research Council}{May 2016 - Sept 2017} 

{\small Project: ``Technologies for socio-educational inclusion of children with chronic illness''}

{\small Principal Investigators: Dr. Vincenza Benigno, Dr. Guglielmo Trentin}

{\small Law:  art. 22, comma 3, della legge 240/2010, art. 6, comma 2 bis, della L. 27 febbraio 2015 n. 11, di conversione del D.L. 31 dicembre 2014 n. 192}

{\small Note: Originally, the contract was from May, 1\textsuperscript{st} 2016 to April, 30\textsuperscript{th} 2017, but it was extend on April, 24\textsuperscript{th} for a 6-month period starting from April, 30\textsuperscript{th} 2017.}


\resumeSubHeadingListEnd

\section{Teaching Experience}
\resumeSubHeadingListStart

\resumeSubheading {Statistical Pscyhometrics}{Milan, IT}{60-hour course, 8 c.f.u.}{\small{Since A.Y. 2021/2022}}

{\small Topics: Introduction to Measurement Theory in Psychology, univariate and bivariate statistics, Probability Theory, Statistical Inference}

{\small School: Bachelor Degree in Psychology, Catholic University of the Sacred Heart, Milan, IT} 

\resumeSubheading {cou\texttt{R}se: An introduction to \texttt{R}}{Milan, IT}{12-hour course}{\small{June 7\textsuperscript{th}-8\textsuperscript{th}, 2023}}

{\small Topics: Variables, functions, and operators, data structures, working with data frames, descriptive statistics, graphics (base R and \texttt{ggplot2})}

{\small School: Graduate School in Psychology, Catholic University of the Sacred Heart, Milan,  IT} 

{\small Further information: Lesson at the Graduate School in Psychology of the Catholic University of the Sacred Heart. }

{\small Course website:  \href{https://github.com/OttaviaE/coRso}{https://github.com/OttaviaE/coRso}} 


\resumeSubheading {\texttt{RMarkdown}: Reproducible analysis, presentations, reports and beyond}{Padova, IT}{20-hour course}{\small{ 2022,  2023}}

{\small Topics: Basics of \texttt{RMarkdown}, integrating code, analysis, and text in one file, presentations with basic \texttt{RMarkdown}, \texttt{xaringan}, and \texttt{beamer}, introduction to \texttt{quarto}}

{\small School: Applied Research Courses Academy, Department of Developmental Psychology and Socialisation, University of Padova, Padova, IT} 

{\small Further information: \href{https://bestr.it/badge/show/3657}{Open-badge} course, Third mission. The course is an intensive school on RMarkdown. So far, it has been delivered three times (June 2022, March 2023, June 2023). } 

{\small Course website: \href{https://arca-dpss.github.io/CorsoRmarkdown/}{https://arca-dpss.github.io/CorsoRmarkdown/}}


\resumeSubheading {Introduction to Item Response Theory with applications in \texttt{R}}{Bolzano, IT}{16-hour course}{December, 19\textsuperscript{th}-21\textsuperscript{st}, 2022}

{\small Topics: Introduction to Item Response Theory for dichotomous responses, model estimation with \texttt{R} on simulated and real data, investigation of Differential Item Functioning. }

{\small School: University of Bolzano, Bolzano, IT} 

{\small Course characteristics: Lesson at the Graduate School in Bressanone-Brixen} 

{\small Course website:  \href{https://ottaviae.github.io/bRixen-IRT/}{https://ottaviae.github.io/bRixen-IRT/}} 

\resumeSubheading {From experiment design to data analysis: How to work with implicit measures}{Padova, IT}{8-hour course}{2022, 2023}

{\small Topics: Programming experiments in Inquisit, Saving data, Import and analyze data with \texttt{R}}

{\small School: II Level Master in Quantitative Psychology, University of Padova, IT} 

{\small Further information: The course in an intensive school on programming and collecting data in Inquisit, and analyze the collected data. So far, it has been delivered twice times (July 2022, July 2023). } 



\resumeSubheading {Programming in R and introduction to shiny}{Padova, IT}{12-hour course}{July, 2022}

{\small Topics: Introduction to \texttt{R} and \texttt{shiny}, Basic and advanced programming in \texttt{shiny}} 

{\small School: Higher Education
	Learning Platform for Quantitative Thinking (QHELP)} 
	
{\small Course website: \href{https://ottaviae.github.io/RcouRse/}{https://ottaviae.github.io/RcouRse/}}

\resumeSubheading {Implicit assessment in psychology: Myths, doubts, and practical applications}{Milano, IT}{2-hour seminar}{June, 10\textsuperscript{th}, 2022}

{\small Topics: Excursus on the concept of implicit in psychological assessment, implicit measures, Psychometrically sound approaches for analyzing implicit measures data} 

{\small School: Methodological School (FoRME), Catholic University of the Sacred Heart, Milan, IT} 

{\small Further information: Third mission.  } 


\resumeSubheading {Psychometrics -- Tutor}{Padova, IT}{30 hours (Second semester)}{Since Academic year 2019/2020}

{\small Topics: Statistical inference, Statistical testing in Psychology} 

{\small School: Bachelor in Psychology, University of Padova, IT} 

\resumeSubheading {The Rasch model: Practical applications in \texttt{R}}{Padova, IT}{6-hour course}{Academic year: 2017/2018 }

{\small Topics: Introduction to \texttt{R}, Analysis of simulated data with the \texttt{TAM} package in \texttt{R}, \emph{infit} and \emph{outfit} computation and interpretation} 

{\small School: Master Degree in Work Psychology, University of Padova, IT} 
\resumeSubHeadingListEnd


\section{Student Supervision}

\resumeSubHeadingListStart

\resumeSubheading 
{Academic Year 2019-2020}{Padova, IT}{University of Padova}
{Bachelor degree in Psychology} 

{\small Thesis: ``Risposta sbagliata!'': L'effetto del feedback sulla performance durante l'Implicit Association Test. 

\small Thesis:  L'effetto del feedback (bult-in correction) nella somministrazione dell'Implicit Association Test 
}


\resumeSubheading 
{Academic Year 2018-2019}{Padova, IT}{University of Padova}
{Bachelor degree in Psychology} 

{\small Thesis: Implicit Association Test, Single Category Implicit Association Test e misure esplicite per la valutazione della preferenza per il cioccolato }
	
	\resumeSubheading 
{Academic Year 2018-2019}{Padova, IT}{University of Padova}
	{Master degree in Social and Work Psychology} 
	
	{\small Thesis: Misure implicite e scelte comportamentali: Confronto tra l’Implicit Association Test e il Single Category Implicit Association Test 
}

	\resumeSubheading 
{Academic Year 2017-2018}{Genova, IT}{University of Genova}
{Bachelor degree in Psychology} 

 {\small Thesis: Validità di costrutto di una misura per la valutazione dell'atteggiamento verso i senzatetto}
 
 	\resumeSubheading 
  {Academic Year 2016-2017}{Padova, IT}{University of Genova}{Bachelor degree in Psychology} 
  
 {\small Thesis: Relazioni fra misure implicite ed esplicite dell’atteggiamento verso le persone tatuate}
 

\resumeSubHeadingListEnd


\section{Skills Summary \& General information}
\resumeSubHeadingListStart

\resumeItem{Awards}{Third position for ``Best presenter' @ Cognitive Science Arena (2020).}

\resumeSubItem{Programming and data analysis}{\texttt{R} (data analysis: 8 years, packages development 4 years), \texttt{shiny} (5 years), \texttt{RMarkdown} (6 years), \LaTeX (\texttt{article}, \texttt{beamer}, \texttt{book}, 5 years), HTML (4 years), CSS (4 Years), Matlab (Basic), Python (Basic), SQL (Basic).}

\resumeSubItem{Software for teaching}{Moodle, Kaltura, Blackboard}

\resumeItem{Languages}{Italian (Mother Tongue), English (Advanced)}

\resumeItem{Personal}{Goal-oriented, team player, strong work ethic, assertive, creative, out-of-the-box thinker.}

\resumeItem{Psicostat}{Member of the Psicostat group since 2020, \href{https://psicostat.dpss.psy.unipd.it/}{https://psicostat.dpss.psy.unipd.it/}}



\resumeSubHeadingListEnd





\section{Advanced training courses}


  \resumeSubHeadingListStart
\resumeSubheading{IRT, CAT, Machine Learning}{Cambridge, UK}{University of Cambridge}
{July 2018}

{\small Topics: \texttt{R}, IRT, Concerto testing platform, CAT, Data mining, sentiment text analysis, Machine learning in practice}

{\small Professor: Dr. Aiden Loe, and Dr. Chris Gibbons}

\resumeSubheading{Tools for Teaching Quantitative Thinking (TquanT)}{Glasgow, UK}{University of Glasgow}
{Feb 2018}

{\small Topics: \texttt{R}, \texttt{shiny}, parameter estimation}

{\small Professor: Dr. Lisa de Bruine, Prof. Francis Tuerlinckx, Prof. Florian Wickelmaier, Prof. Martin Losert, Prof. Hans Colonius}

\resumeSubheading{IRT models}{Florence, IT}{University of Florence}
{Feb - May 2017}

{\small Topics: IRT models for dichotomous and polytomous items, Differential Item Functioning, Goodness of Fit}

{\small Professor: Prof. Caterina Primi, Prof. Francesca Chiesi}


\resumeSubheading{Measurement models in Psychometrics}{Genova, IT}{University of Genova}
{June - July 2016}

{\small Topics: Latent variables modeling - Exploratory Factor Analysis, Confirmatory Factor Analysis - in \texttt{R}}

{\small Professor: Prof. Carlo Chiorri}

\resumeSubheading{Data analysis with \texttt{R}}{Genova, IT}{University of Genova}
{June - July 2015}

{\small Topics: Import data in \texttt{R}, bivariate and multivariate statistical analysis}

{\small Professor: Prof. Carlo Chiorri, Dr. Marcello Passarelli, Dr. Tommaso Piccinno, Dr. Michele Masini}

  \resumeSubHeadingListEnd

%
\section{Internships}
\resumeSubHeadingListStart
\resumeSubheading 
{Trainee in Developmental Psychology}{Genova, IT}{Educational Technologies Institute - National Research Council}
{Sep 2015 - Mar 2016} 

{\small Project ``Technologies for socio-educational inclusion of children with chronic illness''}

{\small Supervisor:  Dr. Vincenza Benigno}

\resumeSubheading 
{Trainee in Psychometrics}{Genova, IT}{University of Genova}
{Mar 2015 - Sep 2015} 

{\small Collaboration with other trainees and Ph.D. students in data analysis}

{\small Supervisor:  Prof. Carlo Chiorri}


\resumeSubHeadingListEnd








%-----------EXPERIENCE-----------------
%\section{Experience}
%  \resumeSubHeadingListStart
%    \resumeSubheading
%    {VMware}{Palo Alto, CA}
%    {Member Of Technical Staff }{Feb 2017 -  Current}
%    \resumeItemListStart
%        \resumeItem{Events and Alert Manager}
%          {Network Fabric Controller is a logically centralized software controller to manage a distributed physical network fabric or a physical network underlay. Designed and developed a library which can be used by any services within Network Fabric Controller to generate events and raise alerts for NFC managed objects. The events and alerts are displayed on the NFC dashboard.}
%          \resumeItem{Upgrade NFC}
%          {Designed and developed an over-the-air and air-gapped upgrade mechanism that is used to upgrade the single node Network Fabric Controller cluster.}
%          \resumeItem{Health Monitoring System}{Designed and developed a monitoring service which is responsible for monitoring the health of all the micro services running inside NFC cluster.}
%          \resumeItem{CLI framework}{Developed an internal command line interface tool which provides a set of commands specific to Network Fabric Controller projects to get the system health, logs and current resource utilization. It can be easily extended to perform various other actions.}
%          \resumeItem{Bootstrap NFC}{Network Fabric Controller is composed of several micro services deployed on the Kubernetes pods on a single-node cluster. Designed and implemented the bootstrapping mechanism to package all the services and deploy on the Kubernetes environment.}
%          \resumeItem{Install/Upgrade/Uninstall NSX agent}{Worked on install, upgrade and uninstall mechanism of NSX agent on workload VMs deployed on NSX cross cloud environment.}
%          \resumeItem{AppDiscovery}{Worked on application profiling feature which provides visualization and details of which processes inside a workload VM are communicating on the network.}
%      \resumeItemListEnd
%      
%    \resumeSubheading
%		{Stony Brook University}{Stony Brook, NY}
%		{Research Assistant - Prof. Erez Zadok }{May 2016 -  August 2016}
%		\resumeItemListStart
%        \resumeItem{System Call Trace Record/Replay}
%          {Worked on building a trace replayer at system call level to reproduce system call operations that were captured during a specific workload using C, C++, DataSeries. Developed a wrapper class that makes C++ functions callable by strace C code.}
%		\resumeItemListEnd
%
%    \resumeSubheading
%    {Samsung Research Institute}{Noida, India}
%    {Software Developer Engineer}{Jun 2012 - July 2015}
%    \resumeItemListStart
%    \resumeItem{Android File System}{}
%    \begin{description}[font=$\bullet$]
%    \item {Involved in board bring-up activities for Android Smart phones based on Exynos and Broadcom chipsets on Android version 4.3 Jelly Bean to Android 5.0 Lollipop.}
%    \item {Experienced in porting of File System (FAT, EXFAT, SDCARDFS, EXT4) on Samsung mobile’s proprietary platform.}
%    \item {Enhanced performance of smart phones having low RAM by analyzing performance using blktrace and tuning kernel parameters. The code was merged in around 15 smart phones.}
%    \end{description}
%    \resumeItemListEnd
%\resumeSubHeadingListEnd
%
%%-----------PROJECTS-----------------
%\section{Academic Projects}
%\resumeSubHeadingListStart
%\resumeSubItem{Plug board Proxy (Networking)}{Developed a plug board proxy that adds an extra layer of encryption to connections towards TCP services. Clients running on same server connect to pbproxy, which then relays all traffic to actual services. (Mar '16)}
%\resumeSubItem{Asynchronous Work Queue Manager (Kernel Programming)}{Developed a kernel module to serve as an asynchronous work queue manager with configurable worker threads. Implemented netlink sockets to propagate callbacks from kernel to user land and throttling to improve job extraction latency. (Nov '15)}
%\resumeSubItem{Anti-Malware Stackable File System (Kernel Programming)}{Implemented a stackable, anti-malware Linux file system that prevents the existing file system from being corrupted by malware by detecting virus pattern while attempting to open, read and write a file. (Oct '15)}
%\resumeSubItem{File Encryption System Call (Kernel Programming)}{Implemented a system call in Linux kernel, which supports multiple ciphers to encrypt or decrypt an input file.( Sep '15)}
%\resumeSubItem{Peg- Solitaire, Connect Four, Sudoku (Game Development)}{Designed a Peg Solitaire, Connect Four and Sudoku using Iterative Deepening Search, Alpha-beta pruning and Backtracking, MRV and Forward Chaining Artificial Intelligence Algorithms respectively in Python. (Aug '15)}
%\resumeSubHeadingListEnd

%-----------Awards-----------------
\section{Submitted papers/papers under review}
\begin{description}
	
	\item[] Epifania, O.M., Anselmi, P., Robusto, E. (under review). Less is more: A new Item Response Theory-based approach for shortening tests \emph{Psychological Methods} [Second round of review]
	\item[] Epifania, O. M., Anselmi, P. \& Robusto, E., (under review).	A guided tutorial on linear mixed-effects models for the analysis of accuracy and response times in experiments with fully-crossed design \emph{Psychological Methods} [First round of review]
	
	
	\item[2022] 
	\item[] Epifania, O. M., Anselmi, P., \& Robusto, E. (2022). Filling the gap between implicit associations and behavior: A Linear Mixed-Effects Rasch Analysis of the Implicit Association Test. \emph{Methodology, 18}(3), 185-202, doi: \url{https://doi.org/10.5964/meth.7155}
	\item[] Epifania, O. M., Anselmi, P., \& Robusto, E. (2022). Implicit social cognition through the years: The Implicit Association Test at age 21. \emph{Psychology of Consciousness: Theory, Research, and Practice, 9}(3). doi: \url{https://doi.org/10.1037/cns0000305}
	
	\item[2021] 
	\item[] Epifania, O. M., Robusto, E., \& Anselmi, P. (2021). Rasch gone mixed: A mixed model approach to the Implicit Association Test. \emph{TPM: Testing, Psychometrics, Methodology in Applied Psychology, 28}(4). doi: 10.4473/TPM28.4.5
	\item[] Paolini, S., Devita, M. Epifania, O.M., Anselmi, P., Sergi, G., Mapelli, D. \& Coin, A. (2021). Perception of stress and cognitive effciency in older adults with mild and moderate dementia during the COVID-19-related lockdown \emph{Journal of Psychosomatic Research, 149}. doi: https://doi.org/10.1016/j.jpsychores.2021.110584
	\item [2020]
	\item[] Epifania, O. M., Anselmi, P., \& Robusto, E. (2020). Dscoreapp: A shiny web application for the computation of the implicit association test d-score. \emph{Frontiers in Psychology, 10}, 2938. doi: 10.3389/fpsyg.2019.02938
	\item[] Epifania, O. M., Anselmi, P., \& Robusto, E. (2020). A fairer comparison between the Implicit Association Test and the Single Category Implicit Association Test. \emph{Testing, Psychometrics, Methodology in Applied Psychology, 27}(2). doi: http://doi.org/10.4473/TPM27.2.4
\end{description}

\section{Papers in Peer-Reviewed Journals (Scopus, WoS)}
\begin{description}
	\item[2023]

\item[] Epifania, O.M., Anselmi, P., Robusto, E. (2023). Pauci sed boni: An Item Response Theory Approach for Shortening Tests. In: Wiberg, M., Molenaar, D., González, J., Kim, JS., Hwang, H. (eds) \emph{Quantitative Psychology}. IMPS 2022. Springer Proceedings in Mathematics \& Statistics, vol 422. Springer, Cham. \url{https://doi.org/10.1007/978-3-031-27781-8_7}
\item[] Epifania, O. M., Robusto, E., \& Anselmi, P. (2023). Is the performance at the Implicit Association Test sensitive to feedback presentation? A Rasch-based analysis. \emph{Psychological
	Research, 87}(3), 737--759. doi: \href{https://doi.org/10.1007/s00426-022-01703-w}{https://doi.org/10.1007/s00426-022-01703-w}


	\item[2022] 
\item[] Epifania, O. M., Anselmi, P., \& Robusto, E. (2022). Filling the gap between implicit associations and behavior: A Linear Mixed-Effects Rasch Analysis of the Implicit Association Test. \emph{Methodology, 18}(3), 185-202, doi: \url{https://doi.org/10.5964/meth.7155}
\item[] Epifania, O. M., Anselmi, P., \& Robusto, E. (2022). Implicit social cognition through the years: The Implicit Association Test at age 21. \emph{Psychology of Consciousness: Theory, Research, and Practice, 9}(3). doi: \url{https://doi.org/10.1037/cns0000305}

\item[2021] 
\item[] Epifania, O. M., Robusto, E., \& Anselmi, P. (2021). Rasch gone mixed: A mixed model approach to the Implicit Association Test. \emph{TPM: Testing, Psychometrics, Methodology in Applied Psychology, 28}(4). doi: 10.4473/TPM28.4.5
\item[] Paolini, S., Devita, M. Epifania, O.M., Anselmi, P., Sergi, G., Mapelli, D. \& Coin, A. (2021). Perception of stress and cognitive effciency in older adults with mild and moderate dementia during the COVID-19-related lockdown \emph{Journal of Psychosomatic Research, 149}. doi: https://doi.org/10.1016/j.jpsychores.2021.110584
\item [2020]
\item[] Epifania, O. M., Anselmi, P., \& Robusto, E. (2020). Dscoreapp: A shiny web application for the computation of the implicit association test d-score. \emph{Frontiers in Psychology, 10}, 2938. doi: 10.3389/fpsyg.2019.02938
\item[] Epifania, O. M., Anselmi, P., \& Robusto, E. (2020). A fairer comparison between the Implicit Association Test and the Single Category Implicit Association Test. \emph{Testing, Psychometrics, Methodology in Applied Psychology, 27}(2). doi: http://doi.org/10.4473/TPM27.2.4
\end{description}

\section{Other Peer-Reviewed publications}

\begin{description}

	
	\item[2020] 
\item[] Epifania, O.M., Anselmi, P., \& Robusto, E., (2020). Implicit measures with reproducible results: The \texttt{implicitMeasures} package. \emph{Journal of Open Source Software, 5}(52), 2394. doi: 10.21105/joss.02394

\item[] Epifania, O. M., Anselmi, P., \& Robusto, E. (2020). \texttt{implicitMeasures}: Computes the Scores for Different Implicit Measures [Computer software manual]. Retrieved from
\href{https://CRAN.R-project.org/package=implicitMeasures}{https://CRAN.R-project.org/package=implicitMeasures} (R package version 0.2.0)

	\item[2019] 
\item[] Epifania, O. M., Anselmi, P., \& Robusto, E. (2019). DscoreApp: A user-friendly web application for computing the Implicit Association Test D score. \emph{Journal of Open Source Software}, 4(42), 1764, https://doi.org/10.21105/joss.01764

\item[] Epifania, O. M. (2019). DscoreApp. \href{http://fisppa.psy.unipd.it/DscoreApp/}{http://fisppa.psy.unipd.it/DscoreApp/}.  (April 2019)

\end{description}

\section{Conferences}
\begin{description}
	
	\item[2023] 
	
			\item[] Epifania, O. M., Stefanutti, L.,  Anselmi, P., Brancaccio, A. \& de Chiusole, D. (2023). Le misure in psicologia sono significanti?
			Il caso del test della Torre di Londra. Presentation at the Symposium: ``Crisi di replicabilit`a o crisi di validit`a? L’importanza delle
			misure'',   \emph{XXVIII Italian Psychology Association Conference, Experimental Psychology Section}, Lucca (IT), Sep, 18\textsuperscript{th}-20\textsuperscript{th} 2021
	
		\item[] Epifania, O. M., Brancaccio, A., de Chiusole, D., Anselmi, P., \& Stefanutti, L. (2023). mat\texttt{R}iks: An R package for rule-based automatic generation of Raven-like matrices. Presentation at the Symposium: ``New frontiers for the adaptive assessment of executive functions'',   \emph{XXVIII Italian Psychology Association Conference, Experimental Psychology Section}, Lucca (IT), Sep, 18\textsuperscript{th}-20\textsuperscript{th} 2021
	
		\item[] Epifania, O. M., Anselmi, P., \& Robusto, E. (2023). Cut it short:
		A new item response theory-based
		approach for shortening tests. Presentation at the \emph{Meeting of the Association of Applied Statistics},  Bologna (IT), Sep, 6\textsuperscript{th}-8\textsuperscript{th} 2023
	
	\item[] Epifania, O. M., Brancaccio, A., de Chiusole, D., Anselmi, P., \& Stefanutti, L. (2023). mat\texttt{R}iks: An R package for rule-based automatic generation of Raven-like matrices. Presentation at the \emph{Meeting of the European Mathematical Psychology Group},  Amsterdam (NL), Jul, 18\textsuperscript{th}-21\textsuperscript{th} 2023
	
	\item[2022]
	\item[] Epifania, O.M., Anselmi P., \& Robusto, E. (2022). Don't say CAT: New Item Response Theory approaches for developing short test forms. Presentation in the Symposium “Recent Advances in Psychometrics II”, \emph{XXX Conference of the Italian Psychology Association Padova}, Padova (IT) Sep, 27\textsuperscript{th}-30\textsuperscript{th} 2022
	
	\item[] Epifania, O.M. (2022). The Rasch model: Questionnaires and beyond. Presentation in the Symposium “Connecting People and Ideas from Psychology and Statistics: The Psicostat Experience”, \emph{XXX Conference of the Italian Psychology Association Padova}, Padova (IT), Sep, 27\textsuperscript{th}-30\textsuperscript{th} 2022
	
		\item[] Epifania, O.M., Anselmi P., \& Robusto, E. (2022). Pauci sed boni: New Item Response Theory-based procedures for shortening tests. Presentation at the \emph{Meeting of the European Mathematical Psychology Group 2022},  Rovereto (IT), Sep, 5\textsuperscript{th}-7\textsuperscript{th} 2022
	
	\item[] Epifania, O.M., Anselmi P., \& Robusto, E. (2022). Pauci sed moni: An Item Response Theory approach for shortening tests. Poster presentation at the \emph{International Meeting of the Psychometric Society} (IMPS), Bologna (IT), July, 11\textsuperscript{th}-15\textsuperscript{th} 2022
	
	\item[2021] 	
	\item[] Epifania, O.M., Anselmi P., \& Robusto, E. (2021). Risposta sbagliata! L’effetto del feedback sulla performance all’Implicit Association Test [Wrong response! The feedback effect on the performance at the Implicit Association]. Presentation at the \emph{XXVII Italian Psychology Association Conference, Experimental Psychology Section}, Lecce (IT), Sep, 8\textsuperscript{th}-10\textsuperscript{th} 2021
	
		\item[] Epifania, O.M., Anselmi P., \& Robusto, E. (2021). Less is more: Una procedura Item Response Theory per lo sviluppo di forme brevi di test. [Less is more: An Item Response Theory Procedure for shortening tests] Presentation at the \emph{XXVII Italian Psychology Association Conference, Experimental Psychology Section}, Lecce (IT), Sep, 8\textsuperscript{th}-10\textsuperscript{th} 2021
		
		\item[2020] 
		

	
	\item[] 	Epifania, O.M., Anselmi P., \& Robusto, E. (2020). Non separare dopo quello che hai somministrato insieme prima: Un approccio misto per l’analisi congiunta delle misure implicite. [Don't split what you administered together. A mixed model approach for the concurrent analysis of implicit measures] Presentation at the \emph{XXVI Italian Psychology Association Conference, Experimental Psychology Section}, Sep 2\textsuperscript{th}-4\textsuperscript{th} 2020
	
	\item[] 	Epifania, O.M., Anselmi, P., \& Robusto, E. (2020). Scoring the implicit: The \texttt{implicitMeasures} package. Oral poster presentation @ useR!2020 Virtual Conference, July, 7\textsuperscript{th}-9\textsuperscript{th}  2020.  Retrievable @ \href{https://www.youtube.com/watch?v=INa426Ru40Y&list=PL4IzsxWztPdmqml-u7PvYOVLdSvD7kjby&index=5}{https://www.youtube.com/watch?v=INa426Ru40Y\&list=PL4IzsxWztPdmqml-u7PvYOVLdSvD7kjby\&index=5}
	
	\item[] 	Epifania, O.M., Anselmi, P., \& Robusto, E. (2020). Scoring the Implicit Association Test has never been easier: DscoreApp. Shiny demo presentation @ eRum 2020 Virtual Conference, June, 17\textsuperscript{th}-20\textsuperscript{th} June. Retrievable @ https://github.com/OttaviaE/erum2020program
	
	\item[] 	Epifania, O.M. (2020). Filling the gap between implicit and behavior: A Rasch modeling of the Implicit Association Test. Presentation, \emph{Cognitive Science Arena}, Bressanone-Brixen (IT), Feb, 7\textsuperscript{th}-8\textsuperscript{th} \underline{(Third position ``Best presenter'').}
	
	\item[2019] 
	\item[] 	Epifania, O. M., Anselmi, P., \& Robusto, E. (2019). DscoreApp: Una Shiny Web application per il calcolo del \emph{D} score. [DscoreApp: A shiny web application for the computaion of the \emph{D} score], Presentation at the \emph{XXV Conference of the Italian Psychology Association, Exeperimental Pscyhology, Experimental Psychology Section}, Milano (IT), Sep, 18\textsuperscript{th}-20\textsuperscript{th} 2019
	
	\item[2018] 
	\item[] 	Epifania, O.M., Robusto, E., \& Anselmi, P. (2018). Rasch gone mixed: A mixed model approach to the Implicit Association Test. Oral Presentation, \emph{Meeting of the European Mathematical Psychology Group 2018}, Genova, 30\textsuperscript{th} July – 2\textsuperscript{nd} August 2018.
	
		\item[2017] 
	\item[] 	Panarello, S., Pronesti, A., Epifania, O. M., Priolo, E. et al. (2017). Afachia Pediatrica Monolaterale e Bilaterale: Valutazione delle Modificazioni Refrattive e del Risultato Funzionale. Presentation at the XXXXII Conference of the Oftalmolgic Society, Genova (IT), Oct, 6\textsuperscript{th}-7\textsuperscript{th} 2017
	
	\item[] 	Epifania, O. M. (2017). Now you see me: Personality predictors of tattoos visibility. Presentation at the \emph{Cognitive Science Arena} 2017, Bressanone-Brixen (IT), Feb 17\textsuperscript{th}-18\textsuperscript{th} 2017
	
		\item[2016] 
	\item[] 	Benigno V., Epifania O.M., Fante, C., Ravicchio, F., \& Caruso, G. (2016). Which technological skills and teaching strategies for inclusive education: Synergies and discordances. Presentation at the \emph{International Conference of Education, Research, and Innovation}, Sevilla (E), Nov  14\textsuperscript{th}-16\textsuperscript{th} 2016
	
	\item[] 	Epifania, O.M., \& Chiorri, C. (2016). L’analisi di variabili di conto con sovrabbondanza di zeri: Il caso della relazione tra caratteristiche di personalità e numero di tatuaggi. [Data analysis with zero over dispersion: The relationship between personality characteristics and number of tattoos] Presentation at the \emph{XXII Conference of the Italian Psychology Association, Exeperimental Pscyhology Section}, Roma (IT), Sept 20\textsuperscript{th}- 22\textsuperscript{nd} 2016
	
\end{description}
%-------------------------------------------
\end{document}