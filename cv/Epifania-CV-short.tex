%-------------------------
% Resume in Latex
% Author : Shubhi Rani
% License : MIT
%------------------------

\documentclass[letterpaper,12pt]{article}

\usepackage{latexsym}
\usepackage[empty]{fullpage}
\usepackage{titlesec}
\usepackage{marvosym}
\usepackage[usenames,dvipsnames]{color}
\usepackage{verbatim}
\usepackage{enumitem}
\usepackage[pdftex]{hyperref}
\usepackage{fancyhdr}
\usepackage[utf8]{inputenc}
\usepackage{lastpage}
\pagenumbering{alph}

\pagestyle{fancy}
\fancyhf{} % clear all header and footer fields
\fancyfoot{}
\renewcommand{\headrulewidth}{0pt}
\renewcommand{\footrulewidth}{0pt}
\pagenumbering{arabic}
\cfoot{Page \thepage\ of \pageref{LastPage}}
% Adjust margins
\addtolength{\oddsidemargin}{-0.375in}
\addtolength{\evensidemargin}{.90in}
\addtolength{\textwidth}{1in}
\addtolength{\topmargin}{-.5in}
\addtolength{\textheight}{1in}

\urlstyle{rm}

\raggedbottom
\raggedright
\setlength{\tabcolsep}{0in}

% Sections formatting
\titleformat{\section}{
  \vspace{-3pt}\scshape\raggedright\large
}{}{0em}{}[\color{black}\titlerule \vspace{-5pt}]

%-------------------------
% Custom commands
\newcommand{\resumeItem}[2]{
  \item\small{
    \textbf{#1}{: #2 \vspace{-2pt}}
  }
}

\newcommand{\resumeItemWithoutTitle}[1]{
  \item\small{
    {\vspace{-2pt}}
  }
}

\newcommand{\resumeSubheading}[4]{
  \vspace{-1pt}\item
    \begin{tabular*}{0.97\textwidth}{l@{\extracolsep{\fill}}r}
      \textbf{#1} & #2 \\
      \textit{\small#3} & \textit{\small #4} \\
    \end{tabular*}\vspace{-5pt}
}


\newcommand{\resumeSubItem}[2]{\resumeItem{#1}{#2}\vspace{-4pt}}

\renewcommand{\labelitemii}{$\circ$}

\newcommand{\resumeSubHeadingListStart}{\begin{itemize}[leftmargin=*]}
\newcommand{\resumeSubHeadingListEnd}{\end{itemize}}
\newcommand{\resumeItemListStart}{\begin{itemize}}
\newcommand{\resumeItemListEnd}{\end{itemize}\vspace{-5pt}}

%-------------------------------------------
%%%%%%  CV STARTS HERE  %%%%%%%%%%%%%%%%%%%%%%%%%%%%


\begin{document}

%----------HEADING-----------------
\begin{tabular*}{\textwidth}{l@{\extracolsep{\fill}}r}
  \textbf{{\LARGE Ottavia M. Epifania, Ph.D.}} & \\
  \href{https://github.com/OttaviaE}{GitHub: https://github.com/OttaviaE} &  \\
  \href{https://osf.io/profile/}{Open Science Framework: https://osf.io/profile/} & \\
    \href{https://ottaviae.github.io/presentations/}{Website: https://ottaviae.github.io/presentations/} & \\
\end{tabular*}

\section{Current Employment}
\resumeSubHeadingListStart
\resumeSubheading
  {Researcher in Tenure Track }{\href{https://webapps.unitn.it/du/it/Persona/PER0271059/Didattica}{Department of Psychology and Cognitive Science}}{July 2024 - On going}
{University of Trento (it)}
 

  
{\small Field of study: Psychometrics}
    
  

 


\resumeSubheading
{Academic Editor}{PeerJ}{Editorial board: \url{https://peerj.com/academic-boards/editors/?page=99}}
{November 2023 - On going}

{\small }


\resumeSubHeadingListEnd

%-----------EDUCATION-----------------
\section{Education}
\resumeSubHeadingListStart
\resumeSubheading
{Ph.D. in Psychological Sciences; cum laudae}{Padova, IT}{University of Padova}
{Oct 2017 - May 2021}

%{\small Thesis defended on May, 14\textsuperscript{th}, 2021}
%
%{\small Thesis reviewers: Prof. David Andrich, Prof. Francis Tuerlinckx. }
%
%{\small Defense committee: Prof. Luca Andrighetto, Prof. Caterina Primi, Prof. Katie Wolsiefer}

{\small \href{https://ottaviae.github.io/presentations/PhD-Thesis/EpifaniaPhDThesis.pdf}{Thesis}: Inglorious Measures: A Linear Mixed-Effects Model approach for a Rasch analysis of implicit measure accuracy and time responses}

%{\small Advisor: Professor Egidio Robusto \\ Co-advisor: Professor Gianmarco Altoè \\ Advisor (Visiting period at OSU): Professor Paul De Boeck}
%
%{\small Attended courses: Null hypothesis testing, \texttt{R} basics/advanced, Inferential Statistics (NHST), Inferential Statistics (Bayesian), Latent trait modeling, Scientific writing.}



\resumeSubheading
{Visiting research scholar}{Ohio, USA}{The Ohio State University}
{Jan 2019 - May 2019}

{\small Collaboration with Prof. De Boeck (i.e., multilevel data modeling, latent variable modeling)}

\resumeSubHeadingListEnd

% previous work experience 


\section{Selection of Scientific seminars}
\resumeSubHeadingListStart

\resumeSubheading {An introduction to Item Response Theory Models with R}{Rovereto, IT}{6-hour Seminar}{November, 17\textsuperscript{th}, 2023}


{\small School: the University of Trento, Rovereto, IT} 


{\small Course website:  \href{https://ottaviae.github.io/IRTintro/}{https://ottaviae.github.io/IRTintro/}} 


\resumeSubheading {cou\texttt{R}se: An introduction to \texttt{R}}{Milan, IT}{12-hour course}{\small{June 7\textsuperscript{th}-8\textsuperscript{th}, 2023}}


{\small School: Graduate School in Psychology, Catholic University of the Sacred Heart, Milan,  IT} 

%{\small Further information: Lesson at the Graduate School in Psychology of the Catholic University of the Sacred Heart. }

{\small Course website:  \href{https://github.com/OttaviaE/coRso}{https://github.com/OttaviaE/coRso}} 


\resumeSubheading {\texttt{RMarkdown}: Reproducible analysis, presentations, reports and beyond}{Padova, IT}{20-hour course}{\small{2022 - 2024}}


{\small School: Applied Research Courses Academy, Department of Developmental Psychology and Socialisation, University of Padova, Padova, IT} 


{\small Course website: \href{https://arca-dpss.github.io/CorsoRmarkdown/}{https://arca-dpss.github.io/CorsoRmarkdown/}}



\resumeSubheading {From the item perspective: Georg Rasch and Item Response Theory Models}{Padova, IT}{PsicoStat}{April 8\textsuperscript{th}, 2021}

{\small School: PsicoStat, University of Padova, IT} 



\resumeSubheading {Shine bright like an open source app: An introduction to shiny }{Padova, IT}{PsicoStat}{May 29\textsuperscript{th}, 2020}

{\small Topics: Introduction to the \texttt{shiny} package for the development of open souce web application in \texttt{R} with practical examples.} 

{\small School: PsicoStat, University of Padova, IT} 

\resumeSubHeadingListEnd

\section{General information}
\resumeSubHeadingListStart


\resumeSubItem{Programming and data analysis}{\texttt{R} (data analysis: 8 years, packages development 4 years), \texttt{shiny} (5 years), \texttt{RMarkdown} (6 years), \LaTeX (\texttt{article}, \texttt{beamer}, \texttt{book}, 5 years), HTML (4 years), CSS (4 Years), Matlab (Basic), Python (Basic), SQL (Basic).}

\resumeSubItem{Software for teaching}{Moodle, Kaltura, Blackboard}

\resumeItem{Languages}{Italian (Mother Tongue), English (Advanced)}



\resumeItem{Psicostat}{Core member of the Psicostat group, \href{https://psicostat.dpss.psy.unipd.it/}{https://psicostat.dpss.psy.unipd.it/}}

\resumeItem{Third best presenter at the Cognitive Science Arena}{Talk: ``Filling the gap between implicit and behaviors: A Rasch modeling of the Implicit Association Test'' presented at the Cognitive Science Arena in Brixen, 2020}

%\resumeItem{Personal}{Goal-oriented, team player, strong work ethic, assertive, creative, out-of-the-box thinker.}


\resumeSubHeadingListEnd





%








%-----------EXPERIENCE-----------------
%\section{Experience}
%  \resumeSubHeadingListStart
%    \resumeSubheading
%    {VMware}{Palo Alto, CA}
%    {Member Of Technical Staff }{Feb 2017 -  Current}
%    \resumeItemListStart
%        \resumeItem{Events and Alert Manager}
%          {Network Fabric Controller is a logically centralized software controller to manage a distributed physical network fabric or a physical network underlay. Designed and developed a library which can be used by any services within Network Fabric Controller to generate events and raise alerts for NFC managed objects. The events and alerts are displayed on the NFC dashboard.}
%          \resumeItem{Upgrade NFC}
%          {Designed and developed an over-the-air and air-gapped upgrade mechanism that is used to upgrade the single node Network Fabric Controller cluster.}
%          \resumeItem{Health Monitoring System}{Designed and developed a monitoring service which is responsible for monitoring the health of all the micro services running inside NFC cluster.}
%          \resumeItem{CLI framework}{Developed an internal command line interface tool which provides a set of commands specific to Network Fabric Controller projects to get the system health, logs and current resource utilization. It can be easily extended to perform various other actions.}
%          \resumeItem{Bootstrap NFC}{Network Fabric Controller is composed of several micro services deployed on the Kubernetes pods on a single-node cluster. Designed and implemented the bootstrapping mechanism to package all the services and deploy on the Kubernetes environment.}
%          \resumeItem{Install/Upgrade/Uninstall NSX agent}{Worked on install, upgrade and uninstall mechanism of NSX agent on workload VMs deployed on NSX cross cloud environment.}
%          \resumeItem{AppDiscovery}{Worked on application profiling feature which provides visualization and details of which processes inside a workload VM are communicating on the network.}
%      \resumeItemListEnd
%      
%    \resumeSubheading
%		{Stony Brook University}{Stony Brook, NY}
%		{Research Assistant - Prof. Erez Zadok }{May 2016 -  August 2016}
%		\resumeItemListStart
%        \resumeItem{System Call Trace Record/Replay}
%          {Worked on building a trace replayer at system call level to reproduce system call operations that were captured during a specific workload using C, C++, DataSeries. Developed a wrapper class that makes C++ functions callable by strace C code.}
%		\resumeItemListEnd
%
%    \resumeSubheading
%    {Samsung Research Institute}{Noida, India}
%    {Software Developer Engineer}{Jun 2012 - July 2015}
%    \resumeItemListStart
%    \resumeItem{Android File System}{}
%    \begin{description}[font=$\bullet$]
%    \item {Involved in board bring-up activities for Android Smart phones based on Exynos and Broadcom chipsets on Android version 4.3 Jelly Bean to Android 5.0 Lollipop.}
%    \item {Experienced in porting of File System (FAT, EXFAT, SDCARDFS, EXT4) on Samsung mobile’s proprietary platform.}
%    \item {Enhanced performance of smart phones having low RAM by analyzing performance using blktrace and tuning kernel parameters. The code was merged in around 15 smart phones.}
%    \end{description}
%    \resumeItemListEnd
%\resumeSubHeadingListEnd
%
%%-----------PROJECTS-----------------
%\section{Academic Projects}
%\resumeSubHeadingListStart
%\resumeSubItem{Plug board Proxy (Networking)}{Developed a plug board proxy that adds an extra layer of encryption to connections towards TCP services. Clients running on same server connect to pbproxy, which then relays all traffic to actual services. (Mar '16)}
%\resumeSubItem{Asynchronous Work Queue Manager (Kernel Programming)}{Developed a kernel module to serve as an asynchronous work queue manager with configurable worker threads. Implemented netlink sockets to propagate callbacks from kernel to user land and throttling to improve job extraction latency. (Nov '15)}
%\resumeSubItem{Anti-Malware Stackable File System (Kernel Programming)}{Implemented a stackable, anti-malware Linux file system that prevents the existing file system from being corrupted by malware by detecting virus pattern while attempting to open, read and write a file. (Oct '15)}
%\resumeSubItem{File Encryption System Call (Kernel Programming)}{Implemented a system call in Linux kernel, which supports multiple ciphers to encrypt or decrypt an input file.( Sep '15)}
%\resumeSubItem{Peg- Solitaire, Connect Four, Sudoku (Game Development)}{Designed a Peg Solitaire, Connect Four and Sudoku using Iterative Deepening Search, Alpha-beta pruning and Backtracking, MRV and Forward Chaining Artificial Intelligence Algorithms respectively in Python. (Aug '15)}
%\resumeSubHeadingListEnd

%-----------Awards-----------------

\section{\texttt{R} packages}
\begin{description}
	\item[] \textbf{Epifania O.M.}, Anselmi P., Robusto E. (2023). \texttt{shortIRT}: Procedures Based on Item Response Theory Models for the
	Development of Short Test Forms [Computer software manual]. R package version 0.1.2. Retrieved from \href{https://cran.r-project.org/web/packages/shortIRT/index.html}{https://cran.r-project.org/web/packages/shortIRT/index.html}
	
	\item[] Brancaccio, A., \textbf{Epifania, O.M.}, \& de Chiusole, D. (2023). \texttt{matRiks}: Generates Raven-Like Matrices According to Rules [Computer software manual]. R package version 0.1.2. Retrieved from \href{https://cran.r-project.org/web/packages/matRiks/index.html}{thttps://cran.r-project.org/web/packages/matRiks/index.html}
	
	\item[] \textbf{Epifania, O. M.}, Anselmi, P., \& Robusto, E. (2020). \texttt{implicitMeasures}: Computes the Scores for Different Implicit Measures [Computer software manual]. Retrieved from
	\href{https://CRAN.R-project.org/package=implicitMeasures}{https://CRAN.R-project.org/package=implicitMeasures} (R package version 0.2.0) (Google Scholar)
	
	
	\item[] \textbf{Epifania, O. M.} (2019). DscoreApp. \href{http://fisppa.psy.unipd.it/DscoreApp/}{http://fisppa.psy.unipd.it/DscoreApp/}.  (April 2019)
\end{description}

\section{Top 5 Publications}
\begin{description}
%	\item[2024]
	\item[] \textbf{Epifania, O. M.}, Anselmi, P. \& Robusto, E., (2024).	A guided tutorial on linear mixed-effects models for the analysis of accuracy and response times in experiments with fully-crossed design \emph{Psychological Methods}. Advance online publication. doi: \url{https://doi.org/10.1037/met0000708}
	

%	\item[2022] 
	\item[] \textbf{Epifania, O.M.}, Anselmi, P., Robusto, E. (2022). Pauci sed boni: An Item Response Theory Approach for Shortening Tests. In: Wiberg, M., Molenaar, D., González, J., Kim, JS., Hwang, H. (eds) \emph{Quantitative Psychology}. IMPS 2022. Springer Proceedings in Mathematics \& Statistics, vol 422. Springer, Cham. \url{https://doi.org/10.1007/978-3-031-27781-8_7 } (WoS, Scopus, Google Scholar)
	
\item[] \textbf{Epifania, O. M.}, Anselmi, P., \& Robusto, E. (2022). Filling the gap between implicit associations and behavior: A Linear Mixed-Effects Rasch Analysis of the Implicit Association Test. \emph{Methodology, 18}(3), 185-202, doi: \url{https://doi.org/10.5964/meth.7155 } (WoS, Scopus, Google Scholar)

%\item[2021] 
\item[] \textbf{Epifania, O. M.}, Robusto, E., \& Anselmi, P. (2021). Rasch gone mixed: A mixed model approach to the Implicit Association Test. \emph{TPM: Testing, Psychometrics, Methodology in Applied Psychology, 28}(4). doi: 10.4473/TPM28.4.5 (WoS, Scopus, Google Scholar)
%	\item[2020] 
\item[] \textbf{Epifania, O.M.}, Anselmi, P., \& Robusto, E., (2020). Implicit measures with reproducible results: The \texttt{implicitMeasures} package. \emph{Journal of Open Source Software, 5}(52), 2394. doi: \url{https://doi.org/10.21105/joss.02394} (Google Scholar)




\end{description}




%-------------------------------------------
\end{document}