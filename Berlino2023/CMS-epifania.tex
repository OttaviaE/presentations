\documentclass[compress]{beamer}
%\usetheme{Montpellier}
%\usecolortheme{dove}
%\useinnertheme{rounded}
%\useoutertheme{smoothbars}
\usetheme{Montpellier}
\usefonttheme{serif}
\usecolortheme{dove}
\useinnertheme{rounded}
\setbeamersize{text margin left=5mm,text margin right=10mm}
\usepackage{graphicx}
\usepackage{multicol}
\setbeamertemplate{navigation symbols}{}
\setbeamersize{text margin left=5mm,text margin right=5mm} 
\setbeamercolor{subsection name}{fg=white}
\setbeamercolor{section name}{fg=white}
\usepackage{xcolor}
\usepackage{tikz}
\usepackage[absolute,overlay]{textpos}
\usepackage{spot}
\usepackage{tikzsymbols}
\usetikzlibrary{mindmap,calc,patterns,decorations.pathmorphing,decorations.markings, arrows, shapes.arrows, shapes, backgrounds}

  \tikzset{
	invisible/.style={opacity=0},
	visible on/.style={alt={#1{}{invisible}}},
	alt/.code args={<#1>#2#3}{%
		\alt<#1>{\pgfkeysalso{#2}}{\pgfkeysalso{#3}} % \pgfkeysalso doesn't change the path
	},
	every overlay node/.style={
		%draw=black,fill=white,rounded corners,
		anchor=north west, inner sep=0pt,
	},
}

 \title[Randomness and possibilities]{When randomness opens new possibilities: Acknowledging the stimulus sampling variability in Experimental Psychology}
\vspace{-1.5mm}\author[OME, PA, ER]{ Ottavia M. Epifania\textsuperscript{1,2,3}, Pasquale Anselmi\textsuperscript{1}, Egidio Robusto\textsuperscript{1}\\ \texttt{ottavia.epifania@unipd.it}}
\institute[]{\small \textsuperscript{1} University of Padova (IT) \\
\textsuperscript{2} Psicostat Group \\
\textsuperscript{3} Catholic Univerisity of the Sacred Heart, Milan (IT) \\}
\vspace*{-5mm}
\date[CMS Conference]{\footnotesize Conference CMS Berlin \\ December 16}
\titlegraphic{\vspace{-5mm}%
	\includegraphics[width=1.5cm,height=1.5cm,keepaspectratio]{img/unipd.png}%\hspace*{9.75cm}~%
	\includegraphics[width=2cm,height=2cm,keepaspectratio]{img/psicostat.png}%
	\includegraphics[width=1.5cm,height=1.5cm,keepaspectratio]{img/unicatt.png}%
}

\definecolor{back}{RGB}{247, 251, 255}
\definecolor{map}{RGB}{111, 172, 232}
\definecolor{childmap}{RGB}{161, 200, 240}
\definecolor{nodemap}{RGB}{141, 170, 199}
\definecolor{rasch}{rgb}{0.0, 0.33, 0.71}
\definecolor{log}{rgb}{0.0, 0.65, 0.58}
\definecolor{diff}{RGB}{33, 113, 181}
\definecolor{single}{RGB}{106, 81, 163}
\definecolor{comp}{RGB}{35, 99, 70}
\definecolor{inc}{RGB}{191, 13, 43}
\definecolor{highlight}{rgb}{0.45, 0.31, 0.59}
\definecolor{section}{RGB}{51,51,179}
\definecolor{typical}{RGB}{8, 69, 148}
\definecolor{model}{RGB}{74, 20, 134}
\definecolor{orangered2}{RGB}{238,64,0}
\definecolor{royalblue3}{RGB}{58,95,205}
\definecolor{springgreen}{RGB}{0,205,102}
\definecolor{magenta}{RGB}{255,0,255}
\def\tikzoverlay{%
	\tikz[remember picture, overlay]\node[every overlay node]
}%

\AtBeginSubsection{\frame{\subsectionpage}}
 
\setbeamertemplate{subsection page}
{
	\begingroup
	\begin{beamercolorbox}[sep=12pt,center]{section title}
		\usebeamerfont{section title}\insertsection\par
	\end{beamercolorbox}
	\vspace*{5pt}
	\begin{beamercolorbox}[sep=8pt,center]{subsection title}
		\usebeamerfont{subsection title}\insertsubsection\par
	\end{beamercolorbox}
	\endgroup
}

\begin{document}
\begin{frame}[plain]
%    \maketitle
\begin{center}
		\large \bfseries When randomness opens new possibilities: \\ Acknowledging the stimulus sampling variability in Experimental Psychology
\end{center}

\vspace{2.5mm}
\begin{center}
	\textbf{Ottavia M. Epifania}\textsuperscript{1,2,3}, Pasquale Anselmi\textsuperscript{1}, Egidio Robusto\textsuperscript{1}
	
	
	\begin{columns}[T]
		\begin{column}{.33\linewidth}
			\centering 	\includegraphics[width=1.5cm,height=1.5cm,keepaspectratio]{img/unipd.png}
		\end{column}
		\begin{column}{.33\linewidth}
			\centering \includegraphics[width=2.5cm,height=2.5cm,keepaspectratio]{img/psicostat.png}
		\end{column}
		\begin{column}{.33\linewidth}
			\centering  \includegraphics[width=1.5cm,height=1.5cm,keepaspectratio]{img/unicatt.png}
		\end{column}
	\end{columns}
	
	\end{center}
	
	\vspace{1.5mm}
	
	\begin{center}
	\small \textsuperscript{1} University of Padova (IT) \\
	\textsuperscript{2} Psicostat Group, Univeristy of Padova (IT) \\
	\textsuperscript{3} Catholic Univerisity of the Sacred Heart, Milan (IT) \\
\end{center}

\vspace{5mm}



\end{frame}



\section{Introduction}

\subsection{Stimuli are fixed, respondents are random}

\subsection{What if}



\begin{frame}{Generalized linear model for dichotomous responses}
	
	\begin{figure}
		\begin{overprint}
			\onslide<1>\includegraphics[width=0.7\linewidth]{img/baseGLM.pdf}
			\onslide<2>\includegraphics[width=0.7\linewidth]{img/linkGLM.pdf}
		\end{overprint}
	\end{figure}
	
	\onslide<2->
	\tikzoverlay (n1) at (8cm, 5.6cm){%
		\begin{minipage}{0.7\linewidth}
			
			\emph{Logit link function g}
			
			$g(\eta_{ps}) = log\left(\frac{\mu_{ps}}{1 - \mu_{ps}}\right)$
			
			\vspace{5mm}
			
			\emph{Inverse} $g^{-1}$
			
			\vspace{5mm}
			
			$g^{-1} = \frac{exp(\eta_{ps})}{1 + exp(\eta_{ps})} $
			
			\vspace{1.5mm}
		%	where $\eta_{ps} = \theta_p + b_s$
		\end{minipage}
	}; 
	
\end{frame}

\begin{frame}{Random effects and random factors}
	Linear component in a (G)LM: 
	\begin{equation}
		\eta = \beta X,
	\end{equation}
	where $\beta$ indicates the coefficients of the fixed intercept and slope(s), and $X$ is the model-matrix.
	
	Linear components in a (Generalized) Linear Mixed-Effects Model (GLMM): 
	
	\begin{equation}
		\eta = \beta X \, Zd,
	\end{equation}
	where $Z$ is the matrix and $d$ is the vector of the random effects (not parameters!)
	
	\onslide<2->
	
	\vspace{2.5mm}
	\begin{center}
		\emph{Best Linear Unbiased Predictors}
	\end{center}
\end{frame}

\begin{frame}{The Rasch model}
	\begin{equation*}
		P(x_{ps} = 1| \theta_p, b_s) = \dfrac{\exp( \theta_p - b_s)}{1 + \exp(\theta_p - b_s)}
	\end{equation*}
	where: 
	
	$\theta_p$: ability of respondent $p$ (i.e., latent trait level of respondent $p$)\\
	$b_s$: difficulty of stimulus $s$ (i.e., "challenging" power of stimulus $s$)\\
	
	\vspace{2.5mm}
	\onslide<2-> 
	\begin{columns}[T] % align columns
		\begin{column}{.50\linewidth}
			\textcolor{diff}{\rule{\linewidth}{2pt}}
			\begin{center}
				\textcolor{diff}{	\large{{Standard}}}
			\end{center}
			
		\end{column}
		
		\hfill%
		\begin{column}{.50\linewidth}
			\textcolor{single}{\rule{\linewidth}{2pt}}
			\begin{center}
				\textcolor{single}{\large{{GLM}}}	
			\end{center}
			
		\end{column}%
	\end{columns}
	
	\vspace{2.5mm}
	\begin{columns}[T]
		\begin{column}{.50\linewidth}
			\vspace*{2.5mm}
			$P(x_{ps} = 1) = \displaystyle \frac{\exp(\theta_p \spot<3-3>[fill=rasch!50]{-} b_s)}{1 + \exp(\theta_p \spot<3-3>[fill=rasch!50]{-} b_s)}$
		\end{column}
		\hfill
		
		\begin{column}{.50\linewidth}
			
			
			\vspace*{2.5mm}
			$P(x_{ps} = 1) = \displaystyle \frac{\exp(\theta_p \, \spot<3-3>[fill=single!50]{+} \, b_s)}{1 + \exp(\theta_p \, \spot<3-3>[fill=single!50]{+} \, b_s)}$
		\end{column}
	\end{columns}	
	
\end{frame}

\section{Random stimuli in Experimental Psychology}

\subsection{Experiment}
\begin{frame}{The stimuli}

	
	\begin{block}{12 Object stimuli}
	\begin{columns}
		\column{.50\linewidth}
		\centering
			White people faces
		\includegraphics[width=\linewidth]{img/white.png}
		
		\column{.50\linewidth}
		\centering
			Black people faces
		\includegraphics[width=\linewidth]{img/black.png}
	\end{columns}		
	\end{block}
	

	
	\begin{block}{16 Attribute stimuli}
		
	\begin{columns}[T]
	\begin{column}{.50\linewidth}
		\begin{center}
			\color{rasch}Positive attributes
		\end{center}
		
		Good, laughter, pleasure, glory, peace, happy, joy, love
	\end{column}
	
	\begin{column}{.50\linewidth}
		\begin{center}
			\color{alert}Negative attributes
		\end{center}
		
		Evil, bad, horrible, terrible, nasty, pain, failure, hate
	\end{column}
\end{columns}
	\end{block}
	\vspace{1.2mm}
	Participants: 62 (F $=48.39$\%, Age = $24.92\pm2.11$ years)

\end{frame}

\begin{frame}{The task}

\begin{block}{Two experimental conditions}
	\begin{columns}[T]
		\begin{column}{.50\linewidth}
			\begin{center}
				\textbf{White-Good/Black-Bad} (WGBB): \\60 trials
			\end{center}
			
			\centering \includegraphics[width=\linewidth]{img/wgbb.png}
		\end{column}
		
		\begin{column}{.50\linewidth}
			\begin{center}
				\textbf{Black-Good/White-Bad} (BGWB): \\ 60 trials
			\end{center}
			
			\centering \includegraphics[width=\linewidth]{img/bgwb.png}
		\end{column}
	\end{columns}
\end{block}

\end{frame}

\subsection{Models}

\begin{frame}
	\begin{footnotesize}
		The expected response $y$ for the observation $i = 1, \ldots, I$ for respondent $p = 1,\ldots, P$ on stimulus $s = 1,\ldots, S$ in condition $c= 1,\ldots, C$:
	\end{footnotesize}
	\footnotesize
	
	
	\vspace{3mm}
	Model 1:
	\begin{equation*}\label{AccuracyMin}
		y_{i} = \spot<2->[fill =blue!15]{logit^{-1}(\alpha + \beta_c X_c}  + \spot<2->[fill=yellow!20]{\alpha_{p[i]} +  \alpha_{s[i]} + \varepsilon_{i})}
	\end{equation*}
	\begin{centering}
		
		$\alpha_p \sim \mathcal{N}(0, \sigma_p^2)$,
		
	\end{centering}
	
	\begin{centering}
		
		$\alpha_s \sim \mathcal{N}(0, \sigma_s^2)$.
		
	\end{centering}
	
	
	
	Model 2: 
	\begin{equation*}\label{Accuracy5}
		y_{i} = \spot<2->[fill =blue!15]{logit^{-1}(\alpha + \beta_c X_c}  + \spot<2->[fill=yellow!20]{ \alpha_{p[i]} +  \beta_{s[i]}c_{i} + \varepsilon_{i})}
	\end{equation*}
	
	\begin{centering}
		$\alpha_p \sim \mathcal{N}(0, \sigma_p^2)$,
		
	\end{centering}
	
	\begin{centering}
		
		$\beta_s \sim \mathcal{MVN}(0, \Sigma_{sc})$.
		
	\end{centering}
	
	Model 3: 
	
	\begin{equation*}
		y_{i} = \spot<2->[fill =blue!15]{logit^{-1}(\alpha + \beta_c X_c}  + \spot<2->[fill=yellow!20]{\alpha_{s[i]} +  \beta_{p[i]}c_{i} + \varepsilon_{i})}
	\end{equation*}
	
	\begin{centering}
		$\alpha_s \sim \mathcal{N}(0, \sigma_s^2)$,
		
	\end{centering}
	
	\begin{centering}
		
		$\beta_p \sim \mathcal{MVN}(0, \Sigma_{pc})$.
		
	\end{centering}
	
	
	\vspace{1.5mm}
	
	\footnotesize{Accuracy: $\epsilon \sim Logistic (0, \sigma^2)$}
	
		\vspace{1.5mm}
\pause
\begin{columns}[T]
	\begin{column}{.50\linewidth}
		\centering
		
		\spot[fill = blue!15]{Fixed Effects}
		
	\end{column}
		\begin{column}{.50\linewidth}
				\centering
	
		\spot[fill = yellow!20]{Random structure}
	\end{column}
\end{columns}
\end{frame}

\subsection{Results}

\begin{frame}{Model 2 is the least wrong model}
		{\centering{\color{rasch}{Rasch model:}} \\ \small{Model 2} \\}
	\centering
	\includegraphics[width=0.8\linewidth]{img/raceParaccuracy.pdf}
\end{frame}

\begin{frame}{Condition--specific easiness}
	
	\vspace{7mm}
	\begin{columns}
		
		\column{.55\linewidth}
		\centering{\textcolor{comp}{\textsc{Highly contributing stimuli}}}
		
		\vspace{5mm}
		\onslide<1->\includegraphics[width=\textwidth]{img/RaceHigh.pdf}
		
		\column{.55\linewidth}
		\centering{\textcolor{inc}{\textsc{Lowly contributing stimuli}}}
		
		\vspace{5mm}
		\onslide<1->\includegraphics[width=\textwidth]{img/RaceLow.pdf}
	\end{columns}
	\tikzoverlay (n1) at (7.cm, 5.3cm){%
		\begin{minipage}{0.12\linewidth}
			\centering	\includegraphics[width=0.65\linewidth]{img/wm1.jpg}
		\end{minipage}
	}; %
	\tikzoverlay (n2) at (10cm, 5.3cm){%
		\begin{minipage}{0.12\linewidth}
			\centering	\includegraphics[width=0.65\linewidth]{img/bf56.jpg}
		\end{minipage}
	}; %
	%\vspace{\fill}
\end{frame}

\section{Discussion}

\begin{frame}
	\begin{itemize}
	\item Improve generalizability of the results to other sets of stimuli 

\item Control for random variance in the data 

\item Allow for obtaining a Rasch-like parametrization of the data 

\item Possibility of extending the (linear) model to other dependent variables (e.g., response times)
		
	\end{itemize}
\end{frame}
\end{document}
